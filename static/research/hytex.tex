\documentclass[11pt]{article}
\usepackage{hytex,ragged2e}
\usepackage{hyperref}
\palatino

\headstyle{\sf}
\titlehead
\boxhead
\sfbpage

\sloppy

\mysection{subsection}{\sfb}{}
\def\thesubsection{}
\setlength{\presubsectionskip}{0.2cm}
\setsidemargin{2.25cm}
\setmarginparleft{2.8cm}

\newenvironment{commands}{\vspace{-\parskip}\begin{quote}}{\end{quote}\vspace{-\parskip}}
\def\head#1{\msubsection{#1}\addcontentsline{toc}{subsection}{#1}}
\sloppy

\def\contentsname{}

\begin{document}\RaggedRight
\label{start}
\title{\HyTeX\ User's Manual}
\author{Rob J Hyndman}
\date{Version 7.0, \Month\ \Year}
\boxtitle
\hytextitle

\HyTeX\ is a \LaTeX\ style file	 for producing even better
documents. It is a collection of commands that I have found useful
and think other \LaTeX\ users probably will too.  It is intended for
use with \LaTeXe.


\HyTeX\ 7.0 is a trimmed down version of \HyTeX\ 6.0. As more
packages have become available, some of the facilities in \HyTeX\
became obsolete. In particular:\vspace*{-0.3cm}
\biz\itemsep=0pt
\item For compact lists, use the \verb|paralist| package.
\item For variations on description lists, use the \verb|mdwlist| package.
\item For bibliographies, use Bib\TeX\ with the \verb|natbib| package.
\item For bold mathematical symbols, use the \verb|bm| package.
\item For blackboard letters, use the \verb|amsfonts| package.
\item For variations on the verbatim environment, use the \verb|verbatim| package.
\item For better control in typesetting theorems, use the \verb|ntheorem| package.
\item For better control over numbering equations and floats, use the \verb|amsmath| package. (This is pretty useful for lots of other things too.)
\eiz
Other trimming occurred by removing facilities that were hardly ever used.


To use \HyTeX, simply begin your \LaTeX\ document like this:
\begin{commands}
\begin{verbatim}
\documentclass{article}
\usepackage{hytex}

\begin{document}
\end{verbatim}
\end{commands}
There is often no need for any other preamble.	The page dimension commands that
many people include in the preamble are unnecessary as \HyTeX\ does all this
for you.  \HyTeX\ may also be used with the report, book or letter document
styles.

When the \HyTeX\ style file is loaded in this way, a number of
additional commands become available.  Almost everything else in
\LaTeX\ will work exactly as normal.

It is assumed in this manual, that readers are familiar with \LaTeX\ as
described in Kopka and Daly's excellent book ``Guide to \LaTeX''.

If you have any suggestions for inclusion in future versions of \HyTeX, or if
you have found a bug, please let me know at
{\tt Rob.Hyndman@buseco.monash.edu.au}.


Some of the \HyTeX\ commands I have borrowed from similar style
files developed by other \LaTeX ers.

\newpage

\head{Abbreviations}

Some \LaTeX\ commands are typed repeatedly.	 Therefore, I have provided some abbreviations for
commonly used sequences of commands.  Of course, the full commands can still be
used.  The abbreviations are:
\bt{ll}
\verb+\bt+	& \verb+\begin{center}\begin{tabular}+ \\
\verb+\et+	& \verb+\end{tabular}\end{center}+ \\
\verb+\biz+ & \verb+\begin{itemize}+ \\
\verb+\eiz+ & \verb+\end{itemize}+ \\
\verb+\ben+ & \verb+\begin{enumerate}+ \\
\verb+\een+ & \verb+\end{enumerate}+ \\
\et


\head{Blank page}

The command \verb|\blankpage| will give you just that!

\head{Boxed equations}

An equation can be placed in a box using
the command \verb|\eqnbox|:
\vspace*{-0.6cm}

\begin{footnotesize}
\begin{verbatim}
$$\eqnbox{\E\hat f (y|x) - f(y|x) = \sigma_K^2
 \left\{ a^2 \frac{h_X'(x)}{h_X(x)}\frac{\partial f(y|x)}{\partial x} +
		\smallfrac12 a^2 \frac{\partial^2 f(y|x)}{\partial x^2} +
		\smallfrac12 b^2 \frac{\partial^2 f(y|x)}{\partial y^2} \right\}}$$
\end{verbatim}
\end{footnotesize}\vspace{-0.5cm}
produces
$$\eqnbox{\E\hat f (y|x) - f(y|x) = \sigma_K^2
 \left\{ a^2 \frac{h_X'(x)}{h_X(x)}\frac{\partial f(y|x)}{\partial x} +
			\smallfrac12 a^2 \frac{\partial^2 f(y|x)}{\partial x^2} +
			\smallfrac12 b^2 \frac{\partial^2 f(y|x)}{\partial y^2} \right\}}$$

\head{Boxed paragraphs}

To place a paragraph (including displayed equations) in a centered box, use
the command \verb|\boxpar|. For example,\vspace*{-\parskip}

\begin{footnotesize}
\begin{verbatim}
\boxpar{10.5cm}{\textbf{Definition:} If $Y_1,\dots,Y_n$ denote observations
with mean $\bar{Y} = \frac{1}{n}\sum_{i=1}^n Y_i$, then their \textit{sample
standard deviation} is defined as
$$s = \sqrt{\frac{1}{n-1} \sum_{i=1}^n (Y_i-\bar{Y})^2.$$}
\end{verbatim}
\end{footnotesize}\vspace{-0.5cm}
produces
\boxpar{10.5cm}{\textbf{Definition:} If $Y_1,\dots,Y_n$ denote observations
with mean $\bar{Y} = \frac{1}{n}\sum_{i=1}^n Y_i$,
then their \textit{sample standard deviation} is defined
as
$$s = \sqrt{\frac{1}{n-1} \sum_{i=1}^n (Y_i-\bar{Y})^2}.$$}
Note that you must specify the width of the box.

If the \verb|color| package is loaded, you can also \verb|\shadebox|
in the same way, but the box will be shaded.


\head{Chapter heads}

Even among well printed books, there is great variation in the format of
section heads.	The font may be bold, sans serif, or small caps, in sizes
ranging from normal to rather large.  If a section number is present, it
may be followed by a period or by a long space.	 The changes here give the
casual \LaTeX er more control over the style of headings.

The format of chapter heads may be controlled by the command
\begin{commands}\begin{verbatim}
\mychapter{<head1>}{<head2>}
\end{verbatim}\end{commands}
Here \verb|<head1>| is the chapter heading to use with the \verb|\chapter|
command (when the chapter is numbered) and \verb|<head2>| is the heading to
use with the \verb|\chapter*| command (when the chapter is not numbered).
The text of the \verb|\chapter| command is referenced in \verb|<head1>| and
\verb|<head2>| by writing \verb|#1|.

For example, the default chapter head is defined by\vspace{-0.4cm}
\begin{verbatim}
  \mychapter{\huge\bf\chapapp\ \thechapter\\[20pt]\Huge\bf #1}
  {\Huge \bf #1}
\end{verbatim}\vspace{-0.4cm}
Here, the command \verb|\chapapp| is usually defined to be ``Chapter'' but
the \LaTeX\ command \verb|\appendix| changes it to ``Appendix''.  The
command \verb|\thechapter| refers to the chapter number.

If you prefer smaller headings in smallcaps without the word ``Chapter''
prefixed, you could try:
\begin{commands}\begin{verbatim}
\mychapter{\LARGE \sc \thechapter. #1}{\LARGE \sc #1}
\end{verbatim}\end{commands}\vspace{\parskip}

There are three inbuilt chapter formats.\\[0.2cm]
\hspace*{1cm}\begin{tabular}{ll}
\verb+\sfchapter+ & Makes chapter headings in sans serif font\\
\verb+\sfbchapter+ & Makes chapter headings in sans serif bold font\\
\verb+\boxchapter+ & Places chapter number in a box
\end{tabular}\\[0.2cm]
Any of these commands can be placed in the preamble.

Chapters always begin on a new page;
\verb|\chaptopsep| is the amount of blank space at the top of the
page before the chapter head, and \verb|\chapaftersep| is the amount of
blank space placed between the chapter head and the text.  They are
initially \verb|1.8cm| and \verb|1.4cm| respectively.  They may be changed
by the \verb|\setlength| command.

An alternative approach is possible using the \verb|fncychap| package.

\head{Columns}

Use the package \verb+multicol+ to set text in columns.	 \HyTeX\ adds a new
command \verb+\newcolumn+ to force a column break.

\head{Date and Time}

To get today's date, use the \TeX\ command \verb+\today+. \HyTeX\ adds an
optional argument to this command to  control the date format and to provide
Australian rather than American formats. The options are given below with
examples of the format.\\[0.2cm]\tab1
\begin{tabular}{ll}
0 (default) & 23 July 1999 \\
1 & 23 Jul 99\\
2 & Friday, 23 July 1999\\
3 & 23.07.99
\end{tabular}\\[0.2cm]
Other formats can be created using the commands \verb|\Dow|, \verb|\Date|,
\verb|\Month|, \verb|\Mon|, \verb|\Year| and \verb|\Yr|.

The current time is available using the new command \verb+\clock+.	For
example, if we type
\begin{commands}\begin{verbatim}
This document was processed at \clock\ on \today.
\end{verbatim}\end{commands}
the printed document will read\\[0.1cm]
\hspace*{1cm}This document was processed at \clock\ on \today.\\[0.1cm]
This is used in the command \verb+\draft+ to add a time stamp to each page.

These commands refer to the date and time at which the dvi file is
produced.


\head{Double spacing}

For theses and journal articles, it is often necessary to use widely
spaced text. This is made possible using the \verb|setspace| package.
The command
\verb+\doublespaced+
in the preamble will give `double-spaced' text
throughout the document.  Single spacing can be turned back on for
parts of the document (such as for tabular) using
\verb|\begin{singlespace}| and \verb|\end{singlespace}| Note that
double spacing is automatically turned off within footnotes and floats
(figures and tables).

The amount of spacing can be controlled via the \verb|\setspacing| command.
The command \verb|\doublespaced|
actually uses \verb|\setspacing{1.75}|.	 If this is too much, try
\verb|\setspacing{1.5}| (or some appropriate value) in the preamble.
\verb|\setspacing| can be used throughout the document to change spacing.

The \verb|newspacing| environment can be used for setting the spacing within a section. e.g.,
\verb|\begin{newspacing}{1.6}| \dots \verb|\end{newspacing}|.

\head{Drafts}

Placing the command \verb+\draft+ in the preamble has the effect of
printing {\sf DRAFT} followed by the time and date at the bottom of
each page. This is useful for keeping track of documents for which
many drafts are printed. The \verb|color| package must be loaded
first.

\head{Figures}

To include graphics, put the command \verb|\usepackage{graphicx}| in
the preamble. The graphs can be pulled in with the command
\begin{commands}\begin{verbatim}
\graph[<placement>]{<options>}{<file>}{<caption>}
\end{verbatim}\end{commands}
The first argument is optional and positions the figure (as with the
table or figure environment). The second argument provides any
optional arguments to the \verb|\includegraphics| command; it can be
left empty but would normally contain something like \verb|width=15cm|.	
The third argument is the name of the file.	 Note the
file extension is not required---it is assumed to be eps if \LaTeX\
is used and jpg or pdf if pdf\LaTeX\ is used. The caption is given
in the fourth argument.

All figures	 are automatically numbered (using the figure counter).
They may be referenced by their file name.	For example \verb|Figure \ref{file}|
will produce the text ``Figure 3'' if it is the third
figure to be included in the document.


\head{Fonts}

\textsfb{Bold sans serif fonts} can be obtained using \verb|\sfb| instead of
\verb|\sf| or \verb|\textsfb| instead of \verb|\textsf|.

PostScript fonts are available using, for example, \verb|\timesroman| in the
preamble.  This changes the roman font to Times-Roman, but leaves the maths, sans
serif and typewriter fonts unchanged.  This is not the same as using the
\verb|times| package which changes
 the roman, sans serif and typewriter
fonts. Similarly, \verb|\bookman|, \verb|\palatino|,
\verb|\newcentury|, \verb|\avantgarde|, \verb|\helvetica|, \verb|\courier|
change to other PostScript fonts.  Several of these commands can be used
together to define san serif, roman and typewriter fonts.

To have both maths and text in palatino font, use the package \verb|mathpazo|.

To obtain page numbers in \textsf{sans serif}, place \verb+\sfpage+ in the
preamble.  For sans serif section headings use \verb+\sfsection+, and for
sans serif chapter headings use \verb+\sfchapter+. Also, \verb|\sfbpage|,
\verb|\sfbsection| and \verb|\sfbchapter| work as above. This document uses
\verb|\sfbsection| and \verb|\sfbpage|.


\head{Headers and footers}
The default pagestyle produces a `running header' containing section names
and page numbers. The headers and footers can be easily modified using the
following commands.

The commands
\begin{commands}\begin{verbatim}
\lhead{<item>}			 \chead{<item>}				 \rhead{<item>}
\end{verbatim}\end{commands}
set the left, center, and right parts of the headers.  The
corresponding commands for footers are
\begin{commands}\begin{verbatim}
\lfoot{<item>}			 \cfoot{<item>}				 \rfoot{<item>}
\end{verbatim}\end{commands}
The first version of these commands was developed by Lance Berc.  They have been
modified for use in \HyTeX.

If the \verb|twoside| style option is invoked before \HyTeX, the macros
switch the right and left items on even numbered pages.	 If you
require different headers and footers for even numbered pages, as
often with books, use the commands
\begin{commands}\begin{verbatim}
\elhead{<item>}			 \echead{<item>}		  \erhead{<item>}
\elfoot{<item>}			 \ecfoot{<item>}		  \erfoot{<item>}
\end{verbatim}\end{commands}
to define them.	 They should follow the commands, \verb|\lhead|,
\verb|chead|, etc.

These commands work with the pagestyle {\tt threepartheadings}	which is
the default.  If no headers and footers are desired, use the command
\verb|\pagestyle{empty}| in the preamble.

To force page headers to be uppercase, use the command
\verb|\uppercasehead|.

The preamble command \verb|\underhead| places a rule of length
\verb|\textwidth| and width \verb|\underheadwidth| a distance of
\verb|\underheadsep| under the header text. The preamble command
\verb|\boxhead| places a box around the header as shown in this document.
Shaded headers are available using \verb|\shadehead| although the \verb|graphicx| package must
be loaded before using this command and it is only available for postscript
printers. For footers use \verb|\overfoot|, \verb|\boxfoot| and
\verb|\shadefoot|.

The font used in headers can be changed using \verb|\headstyle|.  For
example, to make all headers sans serif, put \verb|\headstyle{\sf}| in the
preamble (as was done for this document). For footers, use \verb|\footstyle|.
Any font command included in a \verb|\lhead|, \verb|\rhead|, etc., will
override the \verb|\headstyle|.	 The default is \verb|\headstyle{\sl}|.


\head{Marginal notes}

\verb|\mnote{Comment}| can be used to place text in the margins.
\verb|\setmarginparleft| or \verb|\setmarginparright| should be called first to ensure there is
enough space in the margins.  The text is set in footnotesize.


\head{Mathematical symbols}

Several new commands have been added to produce mathematical symbols:\\[0.3cm]
\begin{tabular}{@{}lp{10.8cm}@{}}
\verb+\invstackrel{}{}+	 & Same as \verb+\stackrel+ but inverted.  E.g.,
				\verb+$\invstackrel{\longrightarrow}{n\rightarrow0}$+\\
				&
				produces $\invstackrel{\longrightarrow}{n\rightarrow0}$\\
\verb+\dist+		 & Produces the `distributed as' symbol, $\dist$ \\
\verb+\smallfrac{}{}+ & Produces small fraction like $\smallfrac{1}{2}$\\
\verb+\bddots+ & Produces backwards diagonal dots: $\bddots$
\end{tabular}


The E symbol for expectation and Pr for probability should appear in roman font rather than maths italics when used
in equations.  This can be easily achieved using \verb+\E+ and \verb+\Pr+.	Similarly, \verb+\var+, \verb+\cov+ and
\verb+\corr+ produce Var, Cov and Corr in maths mode.  To obtain these in sans
serif font use \verb|\sfE| somewhere before you need them.


\head{Page size}

The command \verb+\a4page+ makes the page size A4.	A similar command
\verb|\letterpage| does the same for the US letter size.  A4 is the default.

Three new commands assist with setting the page size.  \verb|\setsidemargin|
describes the space to appear in the margins on each side of the page.	These
are equal in size by default.  The command \verb|\setbinding| adds some space
to the inner margin in case the document is to be bound.  The default
settings have zero binding space and 3cm side margins.	Note these commands
look after the settings for \verb+\oddsidemargin+ and \verb+\evensidemargin+
and take into account whether the page is A4 or US letter.

The command \verb+\setmarginparleft+ is used to set a wide margin on the left
(or odd side) of the page.	The argument is the length of
\verb+\marginparwidth+.	 It also ensures all marginal notes are on the inner side of the page

The command \verb|\setmarginparright| does the same but on the right (or even
side) of the page.


\head{Paragraph indentation}

By default, \HyTeX\ sets paragraph indentation to zero
and the default inter-paragraph space is 1.8 ex.

Customised changes are possible using \verb|\setlength|.


\head{Quotations}
There is a new environment \verb|smallquote| which is identical to
\verb|quote| except that the quotation is set in a small font.

\head{Section heads}
The format of section heads may be controlled by the command
\begin{commands}\begin{verbatim}
\mysection{<sec>}{<style>}{<preface>}
\end{verbatim}\end{commands}
Here {\tt <sec>} is the section level: {\tt section}, {\tt
subsection}, etc.  The style in which the section is to be set is
specified by \verb|<style>|; e.g., \verb|\large\sf|.  The argument
{\tt <preface>} is the text to be set before the text of the section
head\,---\,usually \verb|\thesection| and some punctuation.	 The defaults
are as follows.
\begin{commands}\begin{verbatim}
\mysection{section}{\Large\bf}{\thesection~~}
\mysection{subsection}{\large\bf}{\thesubsection~~}
\mysection{subsubsection}{\normalsize\bf}{\thesubsubsection~~}
\mysection{paragraph}{\normalsize\bf}{\theparagraph~}
\mysection{subparagraph}{\normalsize\bf}{\thesubparagraph~}
\end{verbatim}\end{commands}

To obtain section headings in a sans serif font, include \verb+\sfsection+ in
the preamble.  For bold sans serif font, put \verb|\sfbsection| in the
preamble.

To obtain headings in the left margin, use the command
\verb+\setmarginparleft+ in the preamble, then use \verb|\msection| in place of
\verb|\section| and \verb|\msubsection|	 in place of \verb|\subsection|.  This
is what was done for this document.

The space before and after section headings can be altered by
changing the values of \verb+\presectionskip+ and \verb+\postsectionskip+.
Similarly for subsections.	But note that the space after the section
headings must be positive (I don't know why).

For an alternative approach, use the \verb|sectsty| package.

\head{Table of contents}

In technical articles it is sometimes convenient to place a table of
contents on the first page, right after the title and abstract.	 The
section entries in the default table of contents for the article style tend
to be too widely spaced for this purpose.  The command \verb|\tighttoc|
produces a tighter table of contents.  It was used to generate the table of
contents on the first page.

The command \verb|\tighttoccols|
will produce a table of contents in two columns.  Note that you need to use
the package	 \verb+multicol+ for this to work.

\head{Tables}

Often, you may wish to produce a column of figures aligned on the decimal
point while using {\tt tabular}.  The package \verb|dcolumn| will provide
this facility.

You can set all lines in a table to a certain thickness using a standard \LaTeX\
command.  For example,
\verb+\setlength{arrayrulewidth}{0.5mm}+
sets all lines to have width 0.5mm.
Where you want some lines thicker than others, \HyTeX\ provides two new
commands, \verb+\bhline+ and \verb+\bvline+ to be used in place of \verb+\hline+ and \verb+\vline+.
The next example illustrates the use of
\verb+\bhline+; the use of \verb+\bvline+ is similar.\\[0.3cm]\tab1
\begin{tabular}{l}
 \verb+\begin{tabular}{c}+ \\
 \verb+	 \bhline{4}+\\
 \verb+	 \bf Numbers\\+\\
 \verb+	 \hline+\\
 \verb+	   123\\+ \\
 \verb+	   456\\+ \\
 \verb+	   789\\+ \\
 \verb+	 \bhline{4}+   \\
 \verb+\end{tabular}+
\end{tabular}
\quad which produces \quad
\begin{tabular}{c}
\bhline{4}
\bf Numbers\\
\hline
123\\
456\\
789\\
\bhline{4}
\end{tabular}\\[0.3cm]
The thicker lines are four times the width of the other lines in this case.	 Different multiples are obtained with
different arguments.

\head{Tabs}

A new command \verb|\tab| will insert some hard space providing the
ability to inset text using tabbing, rather like a word processor.
The command takes a numerical argument which gives the distance to
tab in as a multiple of \verb|\tabwidth|.  The default value of
\verb|\tabwidth| is 1cm.  So, for example, \verb|\tab3| will tab in
3cm.


\head{Titles}

The command \verb+\hytextitle+ can be used in place of \verb+\maketitle+ for
the documentstyle `article'.  It takes the arguments from \verb+\title+,
\verb+\author+ and \verb+\date+ but formats them differently from
\verb+\maketitle+. There is also a \verb|\boxtitle| command which, if used
before \verb|\hytextitle|, causes the title to appear in a box. The front
page of this document gives an example of the use of \verb+\boxtitle+ with
\verb|\hytextitle|.

The environment \verb|hytexboxtitle| does a similar thing, except all text in
the environment appears below the title.  This is particularly useful for
abstracts.	The environment \verb|hytextitlepage| is the same as
\verb|hytexboxtitle| except it causes subsequent material to appear on a new
page.

To make the title appear in the header, put \verb|\|\texttt{titlehead} in the
preamble. The font used for
the title and author can be altered using \verb|\titlefont| and
\verb|\authorfont|. (The date is in the same font as the author.)

\head{Two up}

To obtain pages ``two up'' (i.e., two pages printed at half size on each sheet of
paper), use the command \verb|\twoup| in the preamble.	This calls the
package \verb+2up+ and figures out the page dimensions for you.


\end{document}
