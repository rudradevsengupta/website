\documentclass[14pt]{beamer}
%\documentclass[handout,14pt]{beamer}  % OVERHEADS OR HANDOUT

% UNCOMMENT FOLLOWING LINES FOR HANDOUT
%\usepackage{pgfpages}
%\pgfpagesuselayout{2 on 1}[a4paper,border shrink=15mm]

\usetheme{Hytex}

\def\biz{\begin{itemize}[<+-| alert@+>]}
\def\eiz{\end{itemize}}
\def\ben{\begin{enumerate}[<+-| alert@+>]}
\def\een{\end{enumerate}}


\title{Title goes here}
\author{Professor Rob J Hyndman}
\date{}

\begin{document}

\begin{frame}
\titlepage

\centerline{Business \& Economic Forecasting Unit}

\centerline{\includegraphics[width=8cm]{monash1line}}

\centerline{\small\color[rgb]{0.01,0.33,0.58}
\textbf{http://www.robhyndman.info/}}

\end{frame}

\monashlogo

\begin{frame}{Outline}

\tableofcontents
\end{frame}

\section{Introduction}

\begin{frame}{Forecasting functional data}

\biz
\item Observed values are discrete but underlying structures are
continuous functions.

\item Observed values may be noisy but underlying functions are
smooth.

\item {\bf Problem:} To forecast the \textbf{whole function} for
future time periods.
\eiz

\end{frame}


\begin{frame}{Forecasting functional data}

\structure{Some notation}

Let $y_t(x_i)$ be the observed data in period $t$ at location $x_i$,
$i=1,\dots,p$, $t=1,\dots,n$.

\pause

We assume
$$\colorbox[rgb]{1,1,0.4}{$\displaystyle y_t(x_i) =
f_t(x_i) + \sigma_t(x_i)\varepsilon_{t,i}$}
$$
where $\varepsilon_{t,i}$ is iid N(0,1) and $\sigma_t(x_i)$ allows
the amount of noise to vary with $x$.

\pause

\biz
\item We assume $f_t(x)$ is a smooth function of $x$.

\item We need to estimate $f_t(x)$ from the data for $x_1 < x <
x_p$.
\eiz
\end{frame}


\end{document}
